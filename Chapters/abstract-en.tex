%!TEX root = ../template.tex
%%%%%%%%%%%%%%%%%%%%%%%%%%%%%%%%%%%%%%%%%%%%%%%%%%%%%%%%%%%%%%%%%%%%
%% abstract-en.tex
%% NOVA thesis document file
%%
%% Abstract in English([^%]*)
%%%%%%%%%%%%%%%%%%%%%%%%%%%%%%%%%%%%%%%%%%%%%%%%%%%%%%%%%%%%%%%%%%%%

\typeout{NT FILE abstract-en.tex}%

Drug development is a complex process that requires integrating information from many different fields of knowledge.
During the preclinical phase, omics measurements are often used to clarify the phenotypical changes caused by exposure to a therapeutic agent, thereby enabling the reconstruction of the biochemical cascade responsible for the pharmacologic effect (often called Mechanism of Action reconstruction).
This step is crucial in characterizing a therapeutic agent, and researchers rely on numerous computational tools for this process.
However, the abundance of available solutions and the lack of standardized assessment frameworks pose challenges in identifying reliable and efficient tools.
To address this, Clarivate leads the pre-competitive subscription-based, Algorithm Benchmarking Consortium (ABC), which evaluates computational tools for various use cases.
Within the ABC framework, 27 algorithms (13 topology-based, 6 connectivity-based, and 8 enrichment-based algorithms) for identifying key regulators and inferring dysregulated signaling proteins from transcriptomics data were benchmarked to provide an unbiased assessment of available methods.
The algorithms were evaluated together with transcriptomic data derived from 7 databases and molecular networks from OmniPath and  MetaBase\textsuperscript{TM}.
Performance metrics included accuracy in predicting causal regulators, computational scalability, and robustness to input data variations.

Causal reasoning randomWalk and non-causal overconnectivity algorithms showed superior results overall for direct target recovery. Among enrichment-based tools, wmean was the best performer.
At the reference dataset level, MetaBase network lead to the best outcomes. And at the query dataset level, tissue-derived signatures outperformed cell-line-derived ones.
Through this use case, ABC emphasizes the importance of MoA reconstruction as a means to gain a deeper understanding of the pharmacological effects of therapeutic agents. Ultimately, this is essential for characterizing safety profiles, identifying new indications (drug repurposing), exploring potential combinatorial strategies with synergistic effects, and targeting clinically relevant patient subpopulations.

% Palavras-chave do resumo em Ingls
% \begin{keywords}
% Keyword 1, Keyword 2, Keyword 3, Keyword 4, Keyword 5, Keyword 6, Keyword 7, Keyword 8, Keyword 9
% \end{keywords}
\keywords{
  Mechanism of Action \and
  Transcriptomics \and
  Algorithm benchmarking \and
  Causal reasoning
}
