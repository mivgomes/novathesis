%!TEX root = ../template.tex
%%%%%%%%%%%%%%%%%%%%%%%%%%%%%%%%%%%%%%%%%%%%%%%%%%%%%%%%%%%%%%%%%%%%
%% abstract-en.tex
%% NOVA thesis document file
%%
%% Abstract in English([^%]*)
%%%%%%%%%%%%%%%%%%%%%%%%%%%%%%%%%%%%%%%%%%%%%%%%%%%%%%%%%%%%%%%%%%%%

\typeout{NT FILE abstract-en.tex}%

Drug development is a complex process that requires integrating information from many different fields of knowledge.
During the preclinical phase, omics measurements are often used to clarify the phenotypical changes caused by exposure to a therapeutic agent, thereby enabling the reconstruction of the biochemical cascade responsible for the pharmacologic effect (often called Mechanism of Action reconstruction).
This step is crucial in characterizing a therapeutic agent, and researchers rely on numerous computational tools for this process.
However, the abundance of available solutions and the lack of standardized assessment frameworks pose challenges in identifying reliable and efficient tools.
To address this, Clarivate leads the pre-competitive subscription-based Algorithm Benchmarking Consortium (ABC), which evaluates computational tools for various use cases.
Within the ABC framework, several algorithms for identifying key regulators and inferring dysregulated signaling proteins from transcriptomics data (also called "causal reasoning") were benchmarked to provide an unbiased assessment of available methods. This study evaluated 11 algorithms together with transcriptomic data derived from seven databases and molecular networks derived from OmniPath and  MetabaseTM. Performance metrics included accuracy in predicting causal regulators, computational scalability, and robustness to input data variations, offering a deeper understanding of the strengths and limitations of algorithms in this domain. ABC's framework not only facilitates evidence-based algorithm selection for a variety of tasks, ensuring reliable and reproducible results, but also encourages the development of more effective computational workflows in drug discovery by enabling researchers to choose the most suitable tool for their needs. Through this use case, ABC emphasizes the importance of causal regulation as a means to gain a deeper understanding of the pharmacological effects of therapeutic agents. Ultimately, this is essential for characterizing safety profiles, identifying new indications (drug repurposing), exploring potential combinatorial strategies with synergistic effects, and targeting clinically relevant patient subpopulations.

% Palavras-chave do resumo em Ingls
% \begin{keywords}
% Keyword 1, Keyword 2, Keyword 3, Keyword 4, Keyword 5, Keyword 6, Keyword 7, Keyword 8, Keyword 9
% \end{keywords}
\keywords{
  Mechanism of Action \and
  Transcriptomics \and
  Algorithm benchmarking \and
  Causal reasoning
}
