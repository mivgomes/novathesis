%!TEX root = ../template.tex
%%%%%%%%%%%%%%%%%%%%%%%%%%%%%%%%%%%%%%%%%%%%%%%%%%%%%%%%%%%%%%%%%%%%
%% abstract-pt.tex
%% NOVA thesis document file
%%
%% Abstract in Portuguese
%%%%%%%%%%%%%%%%%%%%%%%%%%%%%%%%%%%%%%%%%%%%%%%%%%%%%%%%%%%%%%%%%%%%

\typeout{NT FILE abstract-pt.tex}%

Independentemente da lngua em que a dissertao est escrita, geralmente esta contm pelo menos dois resumos: um resumo na mesma lngua do texto principal e outro resumo numa outra lngua.
%
\begin{verbatim}
    \abstractorder(<MAIN_LANG>):={<LANG_1>,...,<LANG_N>}
\end{verbatim}

Por exemplo, para um documento escrito em Alemo com resumos em Alemo, Ingls e Italiano (por esta ordem), pode usar-se:
\begin{verbatim}
    \ntsetup{abstractorder={de={de,en,it}}}
\end{verbatim}

Relativamente ao seu contedo, os resumos no devem ultrapassar uma pgina e frequentemente tentam responder s seguintes questes ( imprescindvel a adaptao s prticas habituais da sua rea cientfica):


% E agora vamos fazer um teste com uma quebra de linha no hfen a ver se a \LaTeX\ duplica o hfen na linha seguinte se usarmos \verb+"-+ em vez de \verb+-+.
%
% zzzz zzz zzzz zzz zzzz zzz zzzz zzz zzzz zzz zzzz zzz zzzz zzz zzzz zzz zzzz comentar"-lhe zzz zzzz zzz zzzz
%
% Sim!  Funciona! :)

% Palavras-chave do resumo em Portugus
% \begin{keywords}
% Palavra-chave 1, Palavra-chave 2, Palavra-chave 3, Palavra-chave 4
% \end{keywords}
\keywords{
  Primeira palavra-chave \and
  Outra palavra-chave \and
  Mais uma palavra-chave \and
  A ltima palavra-chave
}
% to add an extra black line

