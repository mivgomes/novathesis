%!TEX root = ../template.tex
%%%%%%%%%%%%%%%%%%%%%%%%%%%%%%%%%%%%%%%%%%%%%%%%%%%%%%%%%%%%%%%%%%%%
%% abstract-pt.tex
%% NOVA thesis document file
%%
%% Abstract in Portuguese
%%%%%%%%%%%%%%%%%%%%%%%%%%%%%%%%%%%%%%%%%%%%%%%%%%%%%%%%%%%%%%%%%%%%

\typeout{NT FILE abstract-pt.tex}%

O desenvolvimento de medicamentos é um processo complexo que requer a integração de informações de muitas áreas diferentes do conhecimento.
Durante a fase pré-clínica, dados ómicos são frequentemente utilizados para esclarecer as alterações fenotípicas causadas pela exposição a um agente terapêutico, permitindo assim a reconstrução da cascata bioquímica responsável pelo efeito farmacológico (frequentemente chamada de reconstrução do mecanismo de ação).
Esta etapa é crucial para caracterizar um agente terapêutico, e os investigadores dependem de inúmeras ferramentas computacionais para este processo.
No entanto, a abundância de soluções disponíveis e a falta de estruturas de avaliação padronizadas representam desafios na identificação de ferramentas confiáveis e eficientes.
Com isso em mente, a Clarivate lidera o Algorithm Benchmarking Consortium (ABC), um consórcio baseado em subscrição para farmacêuticas que avalia ferramentas computacionais para vários casos de estudo.
Dentro do ABC, 27 algoritmos (13 baseados em topologia, 6 baseados em conectividade e 8 baseados em enriquecimento) para identificar reguladores-chave e inferir proteínas de sinalização desreguladas a partir de dados transcriptómicos foram comparados para fornecer uma avaliação imparcial dos métodos disponíveis.
Os algoritmos foram avaliados juntamente com dados transcriptómicos derivados de 7 bases de dados e redes moleculares da OmniPath e MetaBase\textsuperscript{TM}.
As métricas de desempenho incluíram precisão na previsão de reguladores causais, escalabilidade computacional e robustez em relação às variações dos dados utilizados pelos algoritmos.

Os algoritmos de raciocínio causal, randomWalk, e não causal, overconnectivity, apresentaram resultados superiores em geral para a recuperação direta de alvos.
Entre as ferramentas baseadas em enriquecimento, o wmean foi o que apresentou melhor desempenho.
A nível de dados de referência, a rede molecular MetaBase apresentou melhores resultados.
E a nível de dados de consulta, as assinaturas derivadas de tecidos superaram as derivadas de linhas celulares.
Através deste caso de uso, o ABC enfatiza a importância da reconstrução do mecanismo de ação de uma perturbação para obter uma melhor compreensão dos efeitos farmacológicos dos agentes terapêuticos.
Em última análise, isso é essencial para caracterizar perfis de segurança, identificar novas indicações (reposicionamento de medicamentos) e explorar estratégias combinatórias potenciais com efeitos sinérgicos.

% Palavras-chave do resumo em Portugus
% \begin{keywords}
% Palavra-chave 1, Palavra-chave 2, Palavra-chave 3, Palavra-chave 4
% \end{keywords}
\keywords{
  Mecanismo de ação \and
  Transcriptómica \and
  Avaliação de algoritmos \and
  Raciocínio causal
}
% to add an extra black line

