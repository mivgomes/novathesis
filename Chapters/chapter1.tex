%!TEX root = ../template.tex
%%%%%%%%%%%%%%%%%%%%%%%%%%%%%%%%%%%%%%%%%%%%%%%%%%%%%%%%%%%%%%%%%%%
%% chapter1.tex
%% NOVA thesis document file
%%
%% Chapter with Introduction
%%%%%%%%%%%%%%%%%%%%%%%%%%%%%%%%%%%%%%%%%%%%%%%%%%%%%%%%%%%%%%%%%%%

\typeout{NT FILE chapter1.tex}%

\makeatletter
\newcommand{\ntifpkgloaded}{%
  \@ifpackageloaded%
}
\makeatother

%%\glsresetall
\chapter{Introduction}
\label{cha:introduction}

\textit{This section expounds the underlying motivation, rationale and goals for the study, emphasizing its significance in the field. It provides context by giving some background on the supporting company and the initiative. Furthermore, it outlies a read’s guide of this thesis.}

\section{Motivation and Goals} % (fold)
\label{sec:motivation_and_goals}


The Research and development (R\&D) of new drugs is a fast-growing area that has also experienced significant growth in complexity in recent years. Due to the time-consuming, costly and multidisciplinary nature of the process, drug discovery remains a challenging domain. Over half of clinical trial failures are attributed to inefficiency, underscoring the importance of identifying and validating pharmacological targets, and highlighting the lack of knowledge of the drug’s mechanism of action (MoA) as one of the major barriers in drug discovery. The thorough understanding of the MoA represents a critical initial step in this process, and computational methods can accelerate this by providing more efficient and cost-effective alternatives to traditional approaches. These approaches can accurately do target identification and prioritization, thereby reducing the need for lengthy experimental trials.

A key to understanding a compound’s MoA lies in transcriptomics data, which captures the molecular changes triggered by a perturbagen and reflects the system’s changes in the gene expression profiles. While traditional RNA sequencing methods remain too costly for large-scale expression signatures, recent high-throughput technological advances, such as the L1000 assay, enable the cost-effective generation and analysis of large-scale omics datasets. Several existing databases provide public access to transcriptomic data from experiments involving diverse chemical and genetic perturbagens across different cell lines. These data can be exploited using various computational tools to establish the causes of specific gene expression changes in a biological system. Three primary approaches have emerged: causal reasoning, connectivity mapping and enrichment tools.

Causal reasoning, a topology-based method, utilizes a list of perturbation signatures and a biological interaction network to determine potential causes for the observed gene expression profile. The network is defined as signed and directed graph describing relations between nodes (e.g., proteins). Efforts to assemble causal molecular relations have increased, resulting in several publicly accessible databases, such as OmniPath, which offers curated prior knowledge networks, however, some causal information remains commercially available, such as MetaBaseTM developed and curated by Clarivate.

The connectivity mapping (CMap) method stems from the efforts to collect and analyze perturbation signatures. It employs similarity scoring to compare a set of known MoA/compound reference signatures with a query gene expression signature resulting from a perturbagen. The principle behind CMap: the higher the similarity between the query and the reference signature, the more likely it is that the mechanism underlying the observed gene expression changes is related to a known perturbation.

On the other hand, enrichment tools take perturbation signatures as query input and utilize prior knowledge, such as regulon network or collections of perturbation-induced differentially expression genes (DEGs), as a reference. The primary function of these tools is to assess whether certain regulons or gene sets (e.g., those associated with transcription factors (TFs)) are significantly enriched in the perturbed data. Several algorithms have been developed based on this approach, each producing an enrichment score.

% section motivation_and_goals (end)

\section{Scope} % (fold)
\label{sec:scope}
This project was conducted within the framework of the Algorithm Benchmarking Consortium (ABC), a subscription-based initiative led by Clarivate for pharmaceutical companies. ABC is dedicated to evaluating a wide range of computational tools for a variety of applications in the life sciences and healthcare field. The topic for this thesis is the development of the ABC’s tenth use case – Causal Regulation – which focuses on benchmark and identify the most optimal tools tailored to specific needs within the drug discovery process by identifying key regulators from transcriptomics data and prior knowledge graphs.

\section{Parallel Contributions} % (fold)
\label{sec:parallel_contributions}

This study expands the state of art in causal reasoning using gene expression data and causal graphs by presenting a robust framework and methods for benchmarking various algorithms designed for this purpose. Beyond this primary focus, several parallel projects with real-world challenges and novel data were developed, and enriched the consultant experience allowing for an expansion in the expertise of bioinformatics. The parallel projects, spanning various domains of computational biology, include:

\begin{enumerate}
\item[\textbf{Skin Microbiome Atlas}] The skin microbiome atlas project involved extensive scientific literature review and dataset curation, followed by a systematic re-analysis of available datasets to ensure consistency and reliability. This was done by pre-processing the raw sequencing data (…understand how deep I can go In terms of details – confidentiality issues)
\item[\textbf{More projects ...}] More projects ...
\end{enumerate}

\section{Structure} % (fold)
\label{sec:structure}

This study is organized in ? chapters. Chapter ~\ref{cha:literature_review} introduces
