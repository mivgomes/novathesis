%!TEX root = ../template.tex
%%%%%%%%%%%%%%%%%%%%%%%%%%%%%%%%%%%%%%%%%%%%%%%%%%%%%%%%%%%%%%%%%%%
%% chapter1.tex
%% NOVA thesis document file
%%
%% Chapter with Introduction
%%%%%%%%%%%%%%%%%%%%%%%%%%%%%%%%%%%%%%%%%%%%%%%%%%%%%%%%%%%%%%%%%%%

\typeout{NT FILE chapter1.tex}%
%%\glsresetall
\chapter{Introduction}
\label{cha:introduction}

\textit{This section expounds the underlying motivation, rationale and goals for the study, emphasizing its significance in the field. It provides context by giving some background on the supporting company and the initiative. Furthermore, it outlies a read’s guide of this thesis.}

\section{Motivation and Goals} % (fold)
\label{sec:motivation_and_goals}


\section{Scope} % (fold)
\label{sec:scope}
This project was conducted within the framework of the Algorithm Benchmarking Consortium (ABC), a subscription-based initiative led by Clarivate for pharmaceutical companies. ABC is dedicated to evaluating a wide range of computational tools for a variety of applications in the life sciences and healthcare field. The topic for this thesis is the development of the ABC’s tenth use case – Causal Regulation – which focuses on benchmark and identify the most optimal tools tailored to specific needs within the drug discovery process by identifying key regulators from transcriptomics data and prior knowledge graphs.

\section{Parallel Contributions} % (fold)
\label{sec:parallel_contributions}

This study expands the state of art in causal reasoning using gene expression data and causal graphs by presenting a robust framework and methods for benchmarking various algorithms designed for this purpose. Beyond this primary focus, several parallel projects with real-world challenges and novel data were developed, and enriched the consultant experience allowing for an expansion in the expertise of bioinformatics. The parallel projects, spanning various domains of computational biology, include:

\begin{enumerate}
\item[\textbf{Skin Microbiome Atlas}] The skin microbiome atlas project involved extensive scientific literature review and dataset curation, followed by a systematic re-analysis of available datasets to ensure consistency and reliability. This was done by pre-processing the raw sequencing data (…understand how deep I can go In terms of details – confidentiality issues)
\item[\textbf{More projects ...}] More projects ...
\end{enumerate}

\section{Structure} % (fold)
\label{sec:structure}

This study is organized in ? chapters. Chapter ~\ref{cha:literature_review} introduces 