%!TEX root = ../template.tex
%%%%%%%%%%%%%%%%%%%%%%%%%%%%%%%%%%%%%%%%%%%%%%%%%%%%%%%%%%%%%%%%%%%%
%% chapter4.tex
%% NOVA thesis document file
%%
%% Chapter with lots of dummy text
%%%%%%%%%%%%%%%%%%%%%%%%%%%%%%%%%%%%%%%%%%%%%%%%%%%%%%%%%%%%%%%%%%%%

\typeout{NT FILE chapter4.tex}%

\glsresetall

\chapter{Adding Support to a New School (work in progress)}
\label{cha:porting_novathesis}



The directory \verb!uminho! contains the customization for all Schools of Universidade do Minho.  This university is an example of the case where the regulations are defined at University level and all the schools apply the same thesis layout and organization.  So, the all the customization is done in the file \verb!uminho/uminho-defaults.ldf!, except the definition of the name and logo of each individual school.

This is the first occurrence of an abbreviation: \gls{CBDD}. 
