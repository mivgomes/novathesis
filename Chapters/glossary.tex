%!TEX root = ../template.tex
%%%%%%%%%%%%%%%%%%%%%%%%%%%%%%%%%%%%%%%%%%%%%%%%%%%%%%%%%%%%%%%%%%%%
%% glossary.tex
%% NOVA thesis document file
%%
%% Glossary definition
%%%%%%%%%%%%%%%%%%%%%%%%%%%%%%%%%%%%%%%%%%%%%%%%%%%%%%%%%%%%%%%%%%%%

\typeout{NT FILE glossary.tex}%

\newglossaryentry{Transcriptomics}{
  name={Transcriptomics},
  sort={Transcriptomics},
  description={The measurement of gene expression levels across the genome (or a subset of genes) obtained from a biological sample. One or more messenger RNA (mRNA) molecules are produced from the transcription of each gene. Transcriptomic technologies use different approaches to measure gene expression by quantifying individual mRNA molecules. Examples of some transcriptomic technologies are microarrays, L1000, and RNA Sequencing (RNA-Seq). Each one of these techniques can generate full gene expression profiles by capturing the expression of all the genes transcribed in a given sample. Transcriptomic analyses are usually performed by sampling at different time points, conditions, or treatments. The raw data can be further processed to identify differentially expressed genes, using metrics such as log₂ fold-change, z-scores, p-values, and q-values.}
}

\newglossaryentry{Gene signature}{
  name={Gene signature},
  sort={Gene signature},
  description={A gene signature is a specific set of genes whose expression pattern is characteristic of a particular biological state, disease outcome, or response to treatment. Usually, it includes the name/ID of the gene, together with a value representing their relative expression (fold-changes or p-values).}
}

\newglossaryentry{Molecular network}{
  name={Molecular network},
  sort={Molecular network},
  description={A graph representation of molecular interactions where nodes represent molecular entities (such as genes and proteins) and edges represent the relationships between them. Networks can be directed (causality), weighted (interaction strength), or signed (activation/inhibition), or without any of these attributes. From a full network, it is possible to extract subsets of interactions (pathways). Pathways are cascades of molecular interactions, and some databases like Kyoto Encyclopedia of Genes and Genomes (KEGG) or Reactome catalog those into specific types, such as metabolic, signaling, or regulatory pathways. Pathway enrichment analysis provides a high-level understanding of the biological processes by identifying coordinated network variations, instead of focusing on individual genes.}
}

\newglossaryentry{Mechanism of Action (MoA)}{
  name={Mechanism of Action (MoA)},
  sort={Mechanism of Action (MoA)},
  description={Molecular cascade by which a perturbation (such as a drug or genetic modification) produces its biological effect. The molecular cascade includes the interaction with the direct molecular target(s) and the immediate downstream cascade of events leading to a certain cellular outcome. The cascade can be reflected in changes in the gene expression.}
}